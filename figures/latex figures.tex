\documentclass[11pt]{article}

\begin{document}

\begin{table}[h!]
\centering
\begin{tabular}{|l|c|}
\hline
\textbf{Institution}                      & \textbf{Number of Cases (approximate)} \\ \hline
Duke University                           & 680                                    \\ \hline
University of California San Francisco    & 600                                    \\ \hline
University of Missouri Columbia           & 400                                    \\ \hline
University of California San Diego        & 350                                    \\ \hline
Heidelberg University Hospital            & 300                                    \\ \hline
University of Michigan                    & 100                                    \\ \hline
Indiana University                        & 70                                     \\ \hline
\textbf{Total}                            & \textbf{2200}                          \\ \hline
\end{tabular}
\caption{Institutional contributions to the BraTS 2024 dataset.}
\label{tab:brats2024_contributions}
\end{table}

\section{Diffusion Processes}

\subsection{Forward Diffusion Process}
The forward diffusion process gradually adds Gaussian noise to the data over \( T \) timesteps, defined as a Markov chain:
\begin{equation}
    q(\mathbf{x}_t \mid \mathbf{x}_{t-1}) = \mathcal{N} \left( \mathbf{x}_t; \sqrt{\alpha_t} \mathbf{x}_{t-1}, (1 - \alpha_t) \mathbf{I} \right)
    \label{eq:forward_step}
\end{equation}
where:
\begin{itemize}[leftmargin=*]
    \item \( \mathbf{x}_0 \) is the original data (e.g., an MRI image).
    \item \( \mathbf{x}_t \) is the data at timestep \( t \).
    \item \( \alpha_t \) is a predefined noise schedule parameter for timestep \( t \).
    \item \( \mathbf{I} \) is the identity matrix.
\end{itemize}

The cumulative forward process can be expressed in closed form as:
\begin{equation}
    q(\mathbf{x}_t \mid \mathbf{x}_0) = \mathcal{N} \left( \mathbf{x}_t; \sqrt{\bar{\alpha}_t} \mathbf{x}_0, (1 - \bar{\alpha}_t) \mathbf{I} \right)
    \label{eq:forward_closed}
\end{equation}
where \( \bar{\alpha}_t = \prod_{s=1}^t \alpha_s \).

\subsection{Reverse Diffusion Process}
The reverse diffusion process aims to remove the added noise, recovering the original data from \( \mathbf{x}_T \) to \( \mathbf{x}_0 \):
\begin{equation}
    p_\theta(\mathbf{x}_{t-1} \mid \mathbf{x}_t) = \mathcal{N} \left( \mathbf{x}_{t-1}; \mu_\theta(\mathbf{x}_t, t), \Sigma_\theta(\mathbf{x}_t, t) \right)
    \label{eq:reverse_step}
\end{equation}
where:
\begin{itemize}[leftmargin=*]
    \item \( \mu_\theta(\mathbf{x}_t, t) \) is the mean predicted by the neural network at timestep \( t \).
    \item \( \Sigma_\theta(\mathbf{x}_t, t) \) is the covariance matrix predicted by the neural network at timestep \( t \).
\end{itemize}

During the reverse process, noise is explicitly subtracted over \( T \) timesteps as follows:
\begin{equation}
    \mathbf{x}_{t-1} = \mu_\theta(\mathbf{x}_t, t) + \Sigma_\theta(\mathbf{x}_t, t) \cdot \epsilon, \quad \epsilon \sim \mathcal{N}(0, \mathbf{I})
    \label{eq:noise_subtraction}
\end{equation}
This iterative denoising continues from \( t = T \) down to \( t = 1 \), ultimately reconstructing the clean data \( \mathbf{x}_0 \).

\subsection{Noise Schedule and Subtraction}
The noise schedule \( \{\alpha_t\} \) controls the amount of noise added at each timestep in the forward process and consequently the noise subtracted in the reverse process. A common choice is a linear or cosine schedule that defines \( \alpha_t \) decreasing over time.

\begin{equation}
    \alpha_t = 1 - \beta_t
    \label{eq:alpha_schedule}
\end{equation}
where \( \beta_t \) is a small positive constant representing the noise variance at timestep \( t \).

\subsubsection*{Example Illustration}
\begin{figure}[h!]
    \centering
    \includegraphics[width=0.8\linewidth]{diffusion_process.png}
    \caption{Illustration of the forward (left) and reverse (right) diffusion processes over \( T \) timesteps. Noise is added incrementally in the forward process and subtracted during the reverse process to reconstruct the original image.}
    \label{fig:diffusion_process}
\end{figure}

\textbf{Given:}
\begin{align*}
\text{Flow Rate (FR)} &= 1.00 \times 10^3 \, \text{ft}^3/\text{min} \\
\text{Time} &= 1 \, \text{day} = 24 \, \text{hours} = 1440 \, \text{minutes}
\end{align*}

\textbf{Find:}
\begin{itemize}
    \item Mass of water in kilograms (kg)
    \item Weight of water in pounds (lb)
\end{itemize}

\textbf{Calculations:}

1. \textbf{Total Volume in ft\textsuperscript{3}:}
\[
\text{Total Volume} = \text{Flow Rate} \times \text{Time} = 1.00 \times 10^3 \, \text{ft}^3/\text{min} \times 1440 \, \text{min} = 1.44 \times 10^6 \, \text{ft}^3
\]

2. \textbf{Convert ft\textsuperscript{3} to Liters (L):}
\[
1 \, \text{ft}^3 = 28.3168 \, \text{L}
\]
\[
\text{Total Volume (L)} = 1.44 \times 10^6 \, \text{ft}^3 \times 28.3168 \, \text{L/ft}^3 \approx 4.08 \times 10^7 \, \text{L}
\]

3. \textbf{Calculate Mass in Kilograms (kg):}
\[
\text{Mass (kg)} = \text{Volume (L)} \times \text{Density of Water} = 4.08 \times 10^7 \, \text{L} \times 1 \, \text{kg/L} = 4.08 \times 10^7 \, \text{kg}
\]

4. \textbf{Convert Mass to Weight in Pounds (lb):}
\[
1 \, \text{kg} = 2.20462 \, \text{lb}
\]
\[
\text{Weight (lb)} = 4.08 \times 10^7 \, \text{kg} \times 2.20462 \, \text{lb/kg} \approx 9.01 \times 10^7 \, \text{lb}
\]

\textbf{Answers:}
\begin{align*}
\text{Mass of water} &= 4.08 \times 10^7 \, \text{kg} \\
\text{Weight of water} &= 9.01 \times 10^7 \, \text{lb}
\end{align*}


1. \textbf{Total Volume in ft\textsuperscript{3}:}
\[
\text{Total Volume} = \text{Flow Rate} \times \text{Time} = 1.00 \times 10^3 \, \text{ft}^3/\text{min} \times 1440 \, \text{min} = 1.44 \times 10^6 \, \text{ft}^3
\]

2. \textbf{Convert ft\textsuperscript{3} to Liters (L):}
\[
1 \, \text{ft}^3 = 28.3168 \, \text{L}
\]
\[
\text{Total Volume (L)} = 1.44 \times 10^6 \, \text{ft}^3 \times 28.3168 \, \text{L/ft}^3 \approx 4.08 \times 10^7 \, \text{L}
\]

3. \textbf{Calculate Mass in Kilograms (kg):}
\[
\text{Mass (kg)} = \text{Volume (L)} \times \text{Density of Water} = 4.08 \times 10^7 \, \text{L} \times 1 \, \text{kg/L} = 4.08 \times 10^7 \, \text{kg}
\]

4. \textbf{Convert Mass to Weight in Pounds (lb):}
\[
1 \, \text{kg} = 2.20462 \, \text{lb}
\]
\[
\text{Weight (lb)} = 4.08 \times 10^7 \, \text{kg} \times 2.20462 \, \text{lb/kg} \approx 9.01 \times 10^7 \, \text{lb}
\]

\textbf{Answers:}
\begin{align*}
\text{Mass of water} &= 4.08 \times 10^7 \, \text{kg} \\
\text{Weight of water} &= 9.01 \times 10^7 \, \text{lb}
\end{align*}

\end{document}
